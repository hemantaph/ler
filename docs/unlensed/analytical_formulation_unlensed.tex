\documentclass[11pt]{article}

% -------------------------------------------------
% Packages
% -------------------------------------------------
\usepackage{amsmath,amssymb}
\usepackage{physics}
\usepackage{graphicx}
\usepackage{booktabs}
\usepackage{longtable}
\usepackage{geometry}
\usepackage{hyperref}
\usepackage{caption}
\usepackage{array}

\geometry{margin=1in}

% -------------------------------------------------
% Metadata
% -------------------------------------------------
\title{Analytical Formulation for Gravitational Wave Event Rates}
\author{Phurailatpam Hemantakumar\\
\href{https://hemantaph.com}{https://hemantaph.com}}
\date{Last updated: June 2024}

% -------------------------------------------------
\begin{document}
\maketitle
\tableofcontents
\newpage

% =================================================
\section{Overview}

This document presents the analytical framework for calculating gravitational-wave (GW) event rates from compact-binary coalescences (CBCs), encompassing binary black holes (BBH), binary neutron stars (BNS), and neutron star--black hole binaries (NSBH). The formulation combines cosmic merger-rate densities with detector sensitivity and selection effects to predict annual detection rates. Practical implementations using the \texttt{ler} package are referenced throughout.

% =================================================
\section{Introduction}

The annual rate of detectable (unlensed) GW events,
$\Delta N^{\rm obs}_{\rm U} / \Delta t$,
represents the expected number of observed CBC mergers per year for a given detector network. It is obtained by combining the total intrinsic merger rate in the detector frame,
$\Delta N_{\rm U} / \Delta t$,
with the population-averaged probability of detection $P({\rm obs})$:
%
\begin{equation}
\begin{split}
\frac{\Delta N^{\rm obs}_{\rm U}}{\Delta t}
=
\frac{\Delta N_{\rm U}}{\Delta t}
\times
P({\rm obs}) \, .
\end{split}
\end{equation}

% =================================================
\section{Parameter-Marginalized Event Rate}

Let the full parameter vector be
$\vec{\theta} = \{\vec{\theta}_{\rm int}, \vec{\theta}_{\rm ext}\}$,
where intrinsic parameters describe the source-frame properties of the binary, and extrinsic parameters describe its configuration relative to the detector network. Explicitly,
%
\begin{equation}
\begin{split}
\vec{\theta} =
\{&
z_s, m_1, m_2, a_1, a_2,
\theta_1, \theta_2, \phi_{12}, \phi_{JL}, \\
&\iota, \phi, \psi, {\rm RA}, {\rm Dec}, t_c
\} .
\end{split}
\end{equation}

Given a joint prior $P(\vec{\theta})$ and conditional detection probability
$P({\rm obs} \mid \vec{\theta})$,
the population-averaged detection probability is
%
\begin{equation}
\begin{split}
P({\rm obs})
=
\int
P({\rm obs} \mid \vec{\theta})\,
P(\vec{\theta})\,
d\vec{\theta} .
\end{split}
\end{equation}

The detectable event rate therefore becomes
%
\begin{equation}
\begin{split}
\frac{\Delta N^{\rm obs}_{\rm U}}{\Delta t}
=
\frac{\Delta N_{\rm U}}{\Delta t}
\int
P({\rm obs} \mid \vec{\theta})\,
P(\vec{\theta})\,
d\vec{\theta} .
\end{split}
\end{equation}

% =================================================
\section{Redshift Distribution and Intrinsic Merger Rates}

The \texttt{ler} framework parameterizes source distance using redshift $z_s$ and assumes $z_s$ is statistically independent of other source parameters. The redshift probability density is proportional to the detector-frame merger-rate density and the comoving volume element:
%
\begin{equation}
\begin{split}
P(z_s)
&\propto
\frac{d^2 N}{dt\, dV_c}
\frac{dV_c}{dz_s} \\
&\propto
\frac{d^2 N}{d\tau\, dV_c}
\frac{d\tau}{dt}
\frac{dV_c}{dz_s} \\
&\propto
\frac{R_{\rm U}(z_s)}{1+z_s}
\frac{dV_c}{dz_s} .
\end{split}
\end{equation}

Here $R_{\rm U}(z_s)$ is the source-frame merger-rate density per comoving volume, and the factor $(1+z_s)^{-1}$ accounts for cosmological time dilation.

Normalizing the distribution introduces
${\cal N}_{\rm U}$,
the total intrinsic merger rate per detector-frame year:
%
\begin{equation}
\begin{split}
P(z_s)
=
\frac{1}{{\cal N}_{\rm U}}
\frac{R_{\rm U}(z_s)}{1+z_s}
\frac{dV_c}{dz_s} ,
\end{split}
\end{equation}
%
with
%
\begin{equation}
\begin{split}
{\cal N}_{\rm U}
=
\int_{z_{\rm min}}^{z_{\rm max}}
\frac{R_{\rm U}(z_s)}{1+z_s}
\frac{dV_c}{dz_s}
\, dz_s .
\end{split}
\end{equation}

The observed event rate can therefore be written compactly as
%
\begin{equation}
\begin{split}
\frac{\Delta N^{\rm obs}_{\rm U}}{\Delta t}
=
{\cal N}_{\rm U}
\left\langle
P({\rm obs} \mid \vec{\theta})
\right\rangle_{\vec{\theta} \sim P(\vec{\theta})} .
\end{split}
\end{equation}

\begin{figure}[t]
\centering
\includegraphics[width=0.8\textwidth]{../_static/Merger_rate_density_and_PDF_of_redshift.png}
\caption{
Redshift evolution of the merger-rate density $R(z_s)$ (left axis) and normalized redshift probability density $P(z_s)$ (right axis) for BBH mergers. The model follows a Madau--Dickinson–like prescription including metallicity and time-delay effects.
}
\label{fig:redshift_pdf}
\end{figure}

% =================================================
\section{Detection Criterion and SNR Modeling}

The conditional detection probability is defined through an SNR threshold $\rho_{\rm th}$:
%
\begin{equation}
\begin{split}
P({\rm obs} \mid \vec{\theta})
\equiv
P_{\rm det}(\vec{\theta}, \rho_{\rm th})
=
\Theta\!\left[
\rho_{\rm obs}(\vec{\theta}) - \rho_{\rm th}
\right] ,
\end{split}
\end{equation}
%
where $\Theta$ is the Heaviside step function.

In practice, the observed SNR $\rho_{\rm obs}$ is treated as a stochastic variable derived from the optimal SNR $\rho_{\rm opt}$. The \texttt{ler} and \texttt{gwsnr} packages adopt a non-central chi-squared model by default, following modern population-analysis practice.

\paragraph{Remark.}
The detection-probability formalism is generic and applies to any signal characterized by an SNR-like detection statistic, including electromagnetic transients.

% =================================================
\section{Simulation Results}

All results shown below are generated using the default configuration of the \texttt{ler} package.

\subsection{Simulation Settings}

A flat $\Lambda$CDM cosmology is assumed with
$H_0 = 70\,{\rm km\,s^{-1}\,Mpc^{-1}}$,
$\Omega_m = 0.3$,
and $\Omega_\Lambda = 0.7$.

Waveforms are generated using the IMRPhenomXPHM approximant with a sampling frequency of $2048\,{\rm Hz}$. The detection threshold is $\rho_{\rm th}=10$, and a $100\%$ duty cycle is assumed.

\subsection{GW Source Parameter Priors}

\begin{longtable}{p{2.2cm} p{1.2cm} p{3.2cm} p{2.4cm} p{4.4cm}}
\caption{Prior distributions for GW source parameters.}
\label{tab:priors}\\
\toprule
Parameter & Unit & Prior & Range & Description \\
\midrule
$z_s$ & -- & Derived & $[0,10]$ & Source redshift \\
$m_{1,2}$ & $M_\odot$ & PowerLaw+Peak / Bimodal & see text & Component masses \\
$a_{1,2}$ & -- & Uniform & $[0,0.99]$ & Spin magnitudes \\
$\theta_{1,2}$ & rad & Sine & $[0,\pi]$ & Spin tilt angles \\
RA & rad & Uniform & $[0,2\pi]$ & Right ascension \\
Dec & rad & Cosine & $[-\pi/2,\pi/2]$ & Declination \\
$\iota$ & rad & Sine & $[0,\pi]$ & Inclination \\
$\psi$ & rad & Uniform & $[0,\pi]$ & Polarization angle \\
\bottomrule
\end{longtable}

\subsection{Detectable vs Intrinsic Populations}

\begin{figure}[t]
\centering
\includegraphics[width=0.8\textwidth]{../_static/Unlensed_Events.png}
\caption{
Corner plot comparing intrinsic (blue) and detectable (orange) BBH populations for the LIGO--Virgo--KAGRA O4 network. Detectable events are biased toward lower redshift and higher masses.
}
\label{fig:corner}
\end{figure}

\subsection{Rate Estimates}

\begin{center}
\begin{tabular}{lccc}
\toprule
Detector Network & BBH yr$^{-1}$ & BNS yr$^{-1}$ & BBH:BNS \\
\midrule
L1--H1--V1 (O4) & 292.7 & 7.4 & 39.5 \\
CE--ET (3G) & 8.87$\times 10^4$ & 1.49$\times 10^5$ & 0.6 \\
\bottomrule
\end{tabular}
\end{center}

\paragraph{Note.}
O4 rates assume design sensitivity and a 100\% duty cycle and should be regarded as optimistic upper limits.

% =================================================
\end{document}
